\documentclass[11pt,preprint]{elsarticle}

\usepackage{lmodern}
%%%% My spacing
\usepackage{setspace}
\setstretch{1.2}
\DeclareMathSizes{12}{14}{10}{10}

% Wrap around which gives all figures included the [H] command, or places it "here". This can be tedious to code in Rmarkdown.
\usepackage{float}
\let\origfigure\figure
\let\endorigfigure\endfigure
\renewenvironment{figure}[1][2] {
    \expandafter\origfigure\expandafter[H]
} {
    \endorigfigure
}

\let\origtable\table
\let\endorigtable\endtable
\renewenvironment{table}[1][2] {
    \expandafter\origtable\expandafter[H]
} {
    \endorigtable
}


\usepackage{ifxetex,ifluatex}
\usepackage{fixltx2e} % provides \textsubscript
\ifnum 0\ifxetex 1\fi\ifluatex 1\fi=0 % if pdftex
  \usepackage[T1]{fontenc}
  \usepackage[utf8]{inputenc}
\else % if luatex or xelatex
  \ifxetex
    \usepackage{mathspec}
    \usepackage{xltxtra,xunicode}
  \else
    \usepackage{fontspec}
  \fi
  \defaultfontfeatures{Mapping=tex-text,Scale=MatchLowercase}
  \newcommand{\euro}{€}
\fi

\usepackage{amssymb, amsmath, amsthm, amsfonts}

\def\bibsection{\section*{References}} %%% Make "References" appear before bibliography


\usepackage[numbers]{natbib}

\usepackage{longtable}
\usepackage[margin=2.3cm,bottom=2cm,top=2.5cm, includefoot]{geometry}
\usepackage{fancyhdr}
\usepackage[bottom, hang, flushmargin]{footmisc}
\usepackage{graphicx}
\numberwithin{equation}{section}
\numberwithin{figure}{section}
\numberwithin{table}{section}
\setlength{\parindent}{0cm}
\setlength{\parskip}{1.3ex plus 0.5ex minus 0.3ex}
\usepackage{textcomp}
\renewcommand{\headrulewidth}{0.2pt}
\renewcommand{\footrulewidth}{0.3pt}

\usepackage{array}
\newcolumntype{x}[1]{>{\centering\arraybackslash\hspace{0pt}}p{#1}}

%%%%  Remove the "preprint submitted to" part. Don't worry about this either, it just looks better without it:
\makeatletter
\def\ps@pprintTitle{%
  \let\@oddhead\@empty
  \let\@evenhead\@empty
  \let\@oddfoot\@empty
  \let\@evenfoot\@oddfoot
}
\makeatother

 \def\tightlist{} % This allows for subbullets!

\usepackage{hyperref}
\hypersetup{breaklinks=true,
            bookmarks=true,
            colorlinks=true,
            citecolor=blue,
            urlcolor=blue,
            linkcolor=blue,
            pdfborder={0 0 0}}


% The following packages allow huxtable to work:
\usepackage{siunitx}
\usepackage{multirow}
\usepackage{hhline}
\usepackage{calc}
\usepackage{tabularx}
\usepackage{booktabs}
\usepackage{caption}


\newenvironment{columns}[1][]{}{}

\newenvironment{column}[1]{\begin{minipage}{#1}\ignorespaces}{%
\end{minipage}
\ifhmode\unskip\fi
\aftergroup\useignorespacesandallpars}

\def\useignorespacesandallpars#1\ignorespaces\fi{%
#1\fi\ignorespacesandallpars}

\makeatletter
\def\ignorespacesandallpars{%
  \@ifnextchar\par
    {\expandafter\ignorespacesandallpars\@gobble}%
    {}%
}
\makeatother


% definitions for citeproc citations
\NewDocumentCommand\citeproctext{}{}
\NewDocumentCommand\citeproc{mm}{%
\href{\#cite.\detokenize{#1}}{#2}\nocite{#1}}

\makeatletter
% allow citations to break across lines
\let\@cite@ofmt\@firstofone
% avoid brackets around text for \cite:
\def\@biblabel#1{}
\def\@cite#1#2{{#1\if@tempswa , #2\fi}}
\makeatother
\newlength{\cslhangindent}
\setlength{\cslhangindent}{1.5em}
\newlength{\csllabelwidth}
\setlength{\csllabelwidth}{3em}
\newenvironment{CSLReferences}[2] % #1 hanging-indent, #2 entry-spacing
{\begin{list}{}{%
	\setlength{\itemindent}{0pt}
	\setlength{\leftmargin}{0pt}
	\setlength{\parsep}{0pt}
	% turn on hanging indent if param 1 is 1
	\ifodd #1
	\setlength{\leftmargin}{\cslhangindent}
	\setlength{\itemindent}{-1\cslhangindent}
	\fi
	% set entry spacing
	\setlength{\itemsep}{#2\baselineskip}}}
{\end{list}}

\usepackage{calc}
\newcommand{\CSLBlock}[1]{\hfill\break\parbox[t]{\linewidth}{\strut\ignorespaces#1\strut}}
\newcommand{\CSLLeftMargin}[1]{\parbox[t]{\csllabelwidth}{\strut#1\strut}}
\newcommand{\CSLRightInline}[1]{\parbox[t]{\linewidth - \csllabelwidth}{\strut#1\strut}}
\newcommand{\CSLIndent}[1]{\hspace{\cslhangindent}#1}


\urlstyle{same}  % don't use monospace font for urls
\setlength{\parindent}{0pt}
\setlength{\parskip}{6pt plus 2pt minus 1pt}
\setlength{\emergencystretch}{3em}  % prevent overfull lines
\setcounter{secnumdepth}{5}

%%% Use protect on footnotes to avoid problems with footnotes in titles
\let\rmarkdownfootnote\footnote%
\def\footnote{\protect\rmarkdownfootnote}
\IfFileExists{upquote.sty}{\usepackage{upquote}}{}

%%% Include extra packages specified by user
\usepackage{booktabs}
\usepackage{longtable}
\usepackage{array}
\usepackage{multirow}
\usepackage{wrapfig}
\usepackage{float}
\usepackage{colortbl}
\usepackage{pdflscape}
\usepackage{tabu}
\usepackage{threeparttable}
\usepackage{threeparttablex}
\usepackage[normalem]{ulem}
\usepackage{makecell}
\usepackage{xcolor}

%%% Hard setting column skips for reports - this ensures greater consistency and control over the length settings in the document.
%% page layout
%% paragraphs
\setlength{\baselineskip}{12pt plus 0pt minus 0pt}
\setlength{\parskip}{12pt plus 0pt minus 0pt}
\setlength{\parindent}{0pt plus 0pt minus 0pt}
%% floats
\setlength{\floatsep}{12pt plus 0 pt minus 0pt}
\setlength{\textfloatsep}{20pt plus 0pt minus 0pt}
\setlength{\intextsep}{14pt plus 0pt minus 0pt}
\setlength{\dbltextfloatsep}{20pt plus 0pt minus 0pt}
\setlength{\dblfloatsep}{14pt plus 0pt minus 0pt}
%% maths
\setlength{\abovedisplayskip}{12pt plus 0pt minus 0pt}
\setlength{\belowdisplayskip}{12pt plus 0pt minus 0pt}
%% lists
\setlength{\topsep}{10pt plus 0pt minus 0pt}
\setlength{\partopsep}{3pt plus 0pt minus 0pt}
\setlength{\itemsep}{5pt plus 0pt minus 0pt}
\setlength{\labelsep}{8mm plus 0mm minus 0mm}
\setlength{\parsep}{\the\parskip}
\setlength{\listparindent}{\the\parindent}
%% verbatim
\setlength{\fboxsep}{5pt plus 0pt minus 0pt}



\begin{document}



\begin{frontmatter}  %

\title{Assessing the Viability of Risk Parity Portfolios in South
African Markets: A Comparative Analysis of Risk-Adjusted Performance}

% Set to FALSE if wanting to remove title (for submission)




\author[Add1]{Vincent Reinshagen}
\ead{28736907@sun.ac.za}





\address[Add1]{Stellenbosch University, Stellenbosch, South Africa}


\begin{abstract}
\small{
This study evaluates the applicability of Risk Parity Portfolios (RPP)
in the South African market, comparing their performance to the Minimum
Variance Portfolio (MVP) and Maximum Diversification Portfolio (MDP).
Utilizing data from 2007 to 2024, the analysis examines risk-adjusted
returns, drawdown metrics, and performance across economic regimes.
While RPPs demonstrate robustness during downturns, their overall
returns lag behind MVP and MDP, particularly in high-growth conditions.
The findings highlight the trade-offs in leveraging RPPs for portfolio
diversification in volatile markets, offering insights for investors
seeking balanced risk allocation strategies.
}
\end{abstract}

\vspace{1cm}





\vspace{0.5cm}

\end{frontmatter}

\setcounter{footnote}{0}



%________________________
% Header and Footers
%%%%%%%%%%%%%%%%%%%%%%%%%%%%%%%%%
\pagestyle{fancy}
\chead{}
\rhead{}
\lfoot{}
\rfoot{\footnotesize Page \thepage}
\lhead{}
%\rfoot{\footnotesize Page \thepage } % "e.g. Page 2"
\cfoot{}

%\setlength\headheight{30pt}
%%%%%%%%%%%%%%%%%%%%%%%%%%%%%%%%%
%________________________

\headsep 35pt % So that header does not go over title




\section{\texorpdfstring{Introduction
\label{Introduction}}{Introduction }}\label{introduction}

Finding the balance between risk and return is a central challenge in
portfolio construction (@
\citeproc{ref-e5a1bb8f-41b7-35c6-95cd-8b366d3e99bc}{Markowitz, 1952}).
Traditional methods for addressing this issue are grounded in Modern
Portfolio Theory (MPT), introduced by Markowitz
(\citeproc{ref-e5a1bb8f-41b7-35c6-95cd-8b366d3e99bc}{1952}), which
provides frameworks such as the Minimum Variance Portfolio (MVP) and the
Maximum Diversification Portfolio (MDP). These approaches focus on risk,
often measured as variance, and aim to optimize either expected returns
or risk minimization (\citeproc{ref-choueifaty2008towards}{Choueifaty,
2008}; \citeproc{ref-e5a1bb8f-41b7-35c6-95cd-8b366d3e99bc}{Markowitz,
1952}). However, an alternative methodology was introduced by Ray Dalio
of Bridgewater Associates in 1996
(\citeproc{ref-bridgewater-2020}{2020}) with the ``All-Weather Fund,''
designed to perform well across all economic conditions and first
assessed academically by Qian \& others
(\citeproc{ref-qian2005risk}{2005}). This fund employs the risk parity
approach (RPP), where assets contribute equally to the portfolio's
overall risk, emphasizing diversification over maximizing returns or
minimizing risk (Maillard, Roncalli \& Tiletche
(\citeproc{ref-maillard2010properties}{2010})). The RPP approach has
demonstrated robust performance in various studies (Maillard \emph{et
al.} (\citeproc{ref-maillard2010properties}{2010}); Chaves, Hsu, Li \&
Shakernia (\citeproc{ref-chaves2011risk}{2011}); Choi, Kim \& Kim
(\citeproc{ref-choi2024diversified}{2024})). However, most research
focuses on portfolios composed of assets from developed markets, which
typically exhibit lower volatility
(\citeproc{ref-loayza2007macroeconomic}{Loayza, Ranciere, Serven \&
Ventura, 2007}). To explore its applicability in more volatile
environments, this paper investigates the performance of an RPP
constructed from South African assets. The analysis compares the risk
and return of the South African RPP to two commonly used risk-minimizing
portfolios, the MVP and the MDP.

This paper addresses the question: \emph{How can the risk parity
approach be effectively implemented to construct a diversified portfolio
of South African assets, and how does it compare to traditional asset
allocation strategies in terms of risk-adjusted returns?}

To answer this question, the next section examines the theoretical
foundations of RPPs. The subsequent section elaborates on the
methodology employed in this study, including the data used, the
characteristics and theory of the benchmark portfolios, and the
performance measures. This is followed by an analysis of the results.
Finally, the paper concludes with insights on the feasibility of
applying the RPP approach to South African assets and outlines
directions for future research.

\section{Theory}\label{theory}

RPPs offer an alternative approach to traditional portfolio construction
by emphasizing equal risk contributions from all assets in a portfolio
(Roncalli (\citeproc{ref-roncalli2013introduction}{2013})). This concept
diverges from the mean-variance optimization framework of MPT, which
focuses on maximizing returns for a given level of risk or minimizing
risk for a desired return level. RPPs are designed to achieve
equilibrium by allocating portfolio weights in a manner that ensures
each asset contributes an equal proportion to the overall portfolio
risk. (\citeproc{ref-asness2012leverage}{Asness, Frazzini \& Pedersen,
2012}; \citeproc{ref-maillard2010properties}{Maillard \emph{et al.},
2010}).

\subsection{Principles of Risk Parity}\label{principles-of-risk-parity}

Qian (\citeproc{ref-qian2011risk}{2011}) highlights the importance of
diversification in constructing robust portfolios, which is the core
idea of RPPs. In MPT, high-return assets often dominate the portfolio
due to their favorable expected risk-return trade-off. This dominance
can lead to concentration in specific asset classes, reducing the
benefits of diversification (\citeproc{ref-qian2011risk}{Qian, 2011}).
To address this issue, thresholds are often imposed on portfolio
composition, such as weight limits for specific assets or industries.
However, these thresholds are based on subjective human estimations,
leading to variability in portfolio performance
(\citeproc{ref-hurst2010understanding}{Hurst, Johnson \& Ooi, 2010}).
RPPs, by contrast, avoids such concentration by equalizing marginal risk
contributions, ensuring that no single asset disproportionately
influences the portfolio's performance. The mathematical foundation of
RPPs involves calculating the marginal risk contribution of each asset
and iteratively adjusting the weights until these contributions are
equalized. This process is mathematically expressed as:

\begin{equation}
RC_i = \frac{\sigma_p}{n}, \quad \forall i
\end{equation}

where RC denotes the risk contribution, which equals the ratio of the
total portfolio standard deviation and the number of assets. The total
portfolio risk is calculated as:

\begin{equation}
\sigma_p = \sqrt{\mathbf{w}^T \boldsymbol{\Sigma} \mathbf{w}}
\end{equation}

This formula aggregates individual asset risks and their covariances,
weighted by portfolio allocation. The risk contribution of asset i is
defined as the product of its portfolio weight w and its marginal risk
contribution to the portfolio:

\begin{equation}
RC_i = w_i \cdot (\boldsymbol{\Sigma} \mathbf{w})_i 
\end{equation}

The optimization problem to achieve risk parity can then be formulated
as:

\begin{equation}
\min_{\mathbf{w}}  \quad \sum_{i=1}^n \left( RC_i - \frac{\sigma_p}{n} \right)^2 
\end{equation}

\subsection{Advantages and
Limitations}\label{advantages-and-limitations}

One primary advantage of RPPs is their robustness across different
market conditions (Hurst \emph{et al.}
(\citeproc{ref-hurst2010understanding}{2010})). Additionally, their
emphasis on risk balance reduces portfolio sensitivity to individual
asset performance, thereby enhancing resilience (Costa \& Kwon
(\citeproc{ref-costa2022data}{2022})). However, RPPs are not without
limitations. The assumption of equal risk contributions can result in
overexposure to low-volatility assets, which may fail to deliver
adequate returns, especially in rising markets
(\citeproc{ref-maillard2010properties}{Maillard \emph{et al.}, 2010}).
Moreover, while RPPs are designed to perform across all economic
regimes, they may fail during structural breaks or extreme shocks, such
as the Covid-19 pandemic (\citeproc{ref-stefanova-2020}{Stefanova,
2020}). This limitation is also shared by other risk-minimizing
portfolios (\citeproc{ref-james2023semi}{James, Menzies \& Chan, 2023}).

\subsection{Empirical Evidence}\label{empirical-evidence}

Empirical research supports the efficacy of the RPP approach in
achieving superior risk-adjusted returns. For example, Asness \emph{et
al.} (\citeproc{ref-asness2012leverage}{2012}) found that RPPs
consistently outperformed traditional asset allocation strategies in
terms of Sharpe ratio. Similarly, Chow, Hsu, Kuo \& Li
(\citeproc{ref-chow2014study}{2014}) demonstrated that RPPs exhibit
lower drawdowns and better performance during economic downturns,
underscoring their utility as an all-weather strategy. Maillard \emph{et
al.} (\citeproc{ref-maillard2010properties}{2010}) showed that RPPs tend
to perform well in various economic environments due to their inherent
diversification. While much of the existing literature focuses on
developed markets, emerging markets present unique challenges and
opportunities. Common characteristics of emerging market assets include
higher volatility, lower liquidity, and greater exposure to
macroeconomic shocks (\citeproc{ref-loayza2007macroeconomic}{Loayza
\emph{et al.}, 2007}).

\section{Methodology}\label{methodology}

The performance of RPPs will be assessed by comparing it to two other
risk-based portfolio construction methods: the MVP and the MDP. The
construction of these portfolios, the data used, and the methods
employed for performance evaluation will be described in the following
subsections.

\subsection{Data}\label{data}

The portfolio will be constructed using the LCL dataset, which contains
a large share of stocks listed on the JSE stock exchange. The portfolio
will cover a time frame from January 2007 to November 2024. This
extended period includes varying economic conditions and two significant
shocks: the World Financial Crisis and the Covid-19 pandemic. The nearly
two-decade observation window provides a robust basis for assessing
long-term performance.

\subsection{Performance Measures}\label{performance-measures}

Although portfolio objectives may vary, a key requirement is
profitability. To evaluate this, cumulative returns will be calculated.
However, given the limitations of cumulative returns---such as their
failure to account for the time period over which returns are
achieved---annualized returns will also be assessed. Annualized returns
provide a time-normalized measure of growth, facilitating comparisons
across investments with different time horizons.

To evaluate risk-adjusted returns and compare the risk levels of the
portfolios, both drawdown and downside risk measures will be used.
Drawdown measures capture the magnitude and duration of declines from an
asset's peak value, providing insights into vulnerability and recovery
during periods of market stress. Conversely, downside risk measures
assess the probability and impact of returns falling below a specified
threshold, emphasizing adverse performance relative to expectations or
benchmarks.

Six downside risk measures will be used for this analysis: the Sharpe
Ratio (SR), the Sortino Ratio (SoR), the Upside Potential Ratio, Value
at Risk (VaR), Conditional Value at Risk (CVaR), and Conditional
Drawdown at Risk (CDD). Since some of these measures require a risk-free
rate or a minimum acceptable return (MAR), the target return will be
based on South African government bonds with a 10-year maturity
(\citeproc{ref-damodaran1999estimating}{Damodaran, 1999}).For drawdown
measures, the Sterling, Calmar, Burke, Pain, and Martin Ratios, along
with the Pain Index and the Ulcer Index, will be calculated.

The main selling point by Bridgewater Associates for their RPP is its
ability to perform well under all economic conditions
(\citeproc{ref-bridgewater-2020}{2020}). To evaluate this claim, South
African Real GDP growth will be used to categorize economic conditions
into three regimes: recession, moderate growth, and high growth. High
growth is defined as GDP growth rates above 3\%, moderate growth as
rates between 0\% and 3\%, and recession as rates below 0\%. These
thresholds are based on South Africa's historical growth rates during
the period under study, with a mean growth rate of 1.5\%.

\subsection{Benchmarks}\label{benchmarks}

To benchmark the performance of the South African RPP, it will be
compared to the MVP and the MDP. The mathematical foundations of these
portfolios are as follows: The MVP minimizes portfolio variance using
the covariance matrix of asset returns and portfolio weights:

\begin{equation}
{Minimize:} \sigma_p^2 = \mathbf{w}^\top \Sigma \mathbf{w}
\end{equation}

The equation represents the portfolio variance, the vector of portfolio
weights, and the covariance matrix of asset returns.\\
The MDP maximizes the diversification ratio, which balances the weighted
average of individual asset volatilities with the portfolio's total
risk:

\begin{equation}
{DR} = \frac{\sum_{i=1}^{N} w_i \sigma_i}{\sqrt{\mathbf{w}^\top \Sigma \mathbf{w}}}
\end{equation}

The RPP, MVP, and MDP represent distinct approaches to portfolio
optimization, all aimed at minimizing risk without relying on expected
returns for construction. The RPP equalizes risk contributions across
assets, focusing on balancing portfolio risk without prioritizing
correlations or volatilities. In contrast, the MVP minimizes total
portfolio variance, often leading to concentrated allocations in
low-risk assets and high sensitivity to asset correlations. The MDP, on
the other hand, maximizes the diversification ratio by allocating more
weight to uncorrelated, high-volatility assets, thereby enhancing
diversification. The shared objective of risk minimization, combined
with their independence from expected returns, makes the MVP and MDP
suitable benchmarks for assessing the performance of the RPP..

\begin{enumerate}
\def\labelenumi{\arabic{enumi}.}
\setcounter{enumi}{4}
\tightlist
\item
  Empirical Results
\end{enumerate}

Figure 3.1 illustrates the cumulative returns of the MDP, the RPP w/o
Bonds, the RPP, and the MVP. The MDP portfolio exhibits the highest
cumulative growth, particularly during market recoveries, suggesting
superior risk-adjusted returns. The MVP and RPP portfolios display lower
but stable growth. The RPP portfolio demonstrates the lowest performance
among all portfolios by a significant margin, as also represented
numerically in Table 3.1.

\begin{figure}[H]

{\centering \includegraphics{Essay_files/figure-latex/Figure1-1} 

}

\caption{Cumulative returns over the whole portfolio runtime \label{Figure1}}\label{fig:Figure1}
\end{figure}

While cumulative returns provide a long-term perspective on portfolio
growth, they fail to capture the nuances of short-term performance and
return variability. Key periods of market stress or exceptional
outperformance are obscured, making it difficult to assess the
consistency and stability of returns. To address this limitation, yearly
rolling returns are analyzed to provide greater granularity, offering
insights into return trends, volatility patterns, and performance
consistency over time.

\begin{table}
\centering
\caption{\label{tab:returns}Cumulative and Annualized Returns}
\centering
\begin{tabular}[t]{l|r|r|r|r}
\hline
  & MVP & RPP w/o Bonds & RPP & MDP\\
\hline
Cumulative Return & 5.813162 & 6.9217365 & 2.9677804 & 9.3606005\\
\hline
Annualized Return & 0.135360 & 0.1467405 & 0.0954672 & 0.1672851\\
\hline
\end{tabular}
\end{table}

Figure 3.2 depicts the annualized rolling returns. While all portfolios
exhibit spikes and dips during significant market events, such as the
2008 financial crisis and the Covid-19 pandemic, the MDP tends to
exhibit higher peaks and faster recoveries, reflecting stronger
performance during recovery periods. The MVP and RPP w/o Bonds
portfolios display more stable but generally lower returns. Again, the
RPP demonstrates the weakest performance, particularly during periods of
high economic growth. The differences in annualized returns between the
portfolios are also highlighted in Table 3.1. Although the returns of
the RPP remain significantly lower, the magnitude of the difference is
reduced for annualized returns.

\begin{figure}[H]

{\centering \includegraphics{Essay_files/figure-latex/Figure2-1} 

}

\caption{Annualized rolling returns \label{Figure2}}\label{fig:Figure2}
\end{figure}

Given the goal of minimizing risk and protecting investors from losses,
the drawdown ratio provides an important tool for understanding the
potential risk an investor may face. Drawdown focuses on the severity
and duration of losses, capturing tail-risk events and periods of
underperformance. By plotting drawdowns, one can visually compare the
resilience of portfolios during market downturns, identify recovery
patterns, and assess risk-adjusted returns over time.

\begin{figure}[H]

{\centering \includegraphics{Essay_files/figure-latex/Figure3-1} 

}

\caption{Drawdowns \label{Figure3}}\label{fig:Figure3}
\end{figure}

\subsection{Drawdown Risk Measures}\label{drawdown-risk-measures}

The visual analysis of drawdowns in Figure 3.3 is inconclusive, as no
portfolio exhibits significant outliers, either positively or
negatively. Therefore, seven drawdown ratios are calculated and
presented in Table 3.2.

The analysis of drawdown ratios reveals substantial differences in the
risk-adjusted performance of the portfolios, highlighting the impact of
drawdown risk on portfolio efficiency. The Sterling Ratio and Calmar
Ratio, which measure returns relative to drawdowns, demonstrate that the
MDP achieves the highest performance (0.3625 and 0.4617, respectively).
This suggests superior risk-adjusted returns compared to the other
portfolios, with the RPP showing the weakest performance in both metrics
(0.1532 and 0.3162, respectively).

The Burke Ratio, which penalizes volatility and accounts for all
drawdowns, further underscores the robustness of the MDP (0.1434), which
far outperforms the RPP (0.0140). The Pain Index and Ulcer Index, which
assess the severity and duration of drawdowns, indicate that the RPP w/o
Bonds experiences the shallowest drawdowns (0.0520 and 0.0904,
respectively), closely followed by the MDP (0.0532 and 0.1018). However,
the higher Pain Ratio of the MDP (1.4840) reflects its ability to
deliver greater returns relative to drawdown risk compared to the RPP
w/o Bonds.

Lastly, the Martin Ratio, which measures returns adjusted for drawdown
risk, further supports the outperformance of the MDP (0.7757). In
contrast, the RPP (0.0755) and other portfolios demonstrate
significantly lower values, emphasizing their vulnerability to
drawdowns. Overall, the MDP consistently excels across all
drawdown-related metrics, underscoring its resilience and efficiency in
managing drawdown risk. In contrast, the RPP emerges as the least
effective portfolio across these measures.

\begin{table}
\centering
\caption{\label{tab:drawdowns-ratio}Drawdowns ratio table}
\centering
\begin{tabular}[t]{l|r|r|r|r}
\hline
  & MVP & RPP w/o Bonds & RPP & MDP\\
\hline
Sterling ratio & 0.2640 & 0.2944 & 0.2376 & 0.3638\\
\hline
Calmar ratio & 0.3280 & 0.3683 & 0.3162 & 0.4649\\
\hline
Burke ratio & 0.0903 & 0.0990 & 0.0140 & 0.1505\\
\hline
Pain index & 0.0657 & 0.0520 & 0.0457 & 0.0542\\
\hline
Ulcer index & 0.1125 & 0.0904 & 0.0757 & 0.1017\\
\hline
Pain ratio & 0.6942 & 1.0960 & 0.1250 & 1.4317\\
\hline
Martin ratio & 0.4052 & 0.6304 & 0.0755 & 0.7622\\
\hline
\end{tabular}
\end{table}

\subsection{Downside Risk Measures}\label{downside-risk-measures}

Building on the analysis of drawdown ratios in Table 3.2, which
emphasized the portfolios' resilience and efficiency in managing
downside risk, Table 3.3 and Table 3.4 shift focus to downside
risk-adjusted performance metrics. These metrics provide a more detailed
perspective on how the portfolios perform under various measures of
risk, further complementing the insights from the drawdown ratios.

\subsubsection{Sharpe Ratios}\label{sharpe-ratios}

The second table displays the Sharpe Ratios of the different portfolios.
The standard deviation (StdDev) Sharpe ratio, which evaluates returns
relative to total volatility, shows that RPP w/o Bonds (-10.79) achieves
the best performance, whereas RPP exhibits the weakest performance
(-15.50). This suggests that the RPP struggles significantly with
overall risk-adjusted returns. The Value at Risk (VaR) Sharpe ratio,
which accounts for the risk of extreme losses, reinforces this trend.
RPP w/o Bonds again outperforms the other portfolios (-7.08), while RPP
demonstrates greater vulnerability to tail risk (-9.67). Similarly, the
Expected Shortfall (ES) Sharpe ratio, which considers the average of
worst-case losses, indicates marginal differences among the portfolios.
RPP w/o Bonds (-3.05) slightly outperforms others, while RPP (-3.67)
consistently underperforms. The SemiSD Sharpe ratio, which uses downside
volatility as the risk measure, further confirms the relative
superiority of RPP w/o Bonds (-10.34) over other portfolios, followed by
the Minimum Variance portfolio (-11.32). In contrast, RPP (-14.64)
exhibits significantly weaker downside risk-adjusted performance,
reflecting heightened sensitivity to negative returns.

The analysis demonstrates that RPP w/o Bonds consistently achieves
better risk-adjusted returns under all downside risk measures, while RPP
lags across all metrics, indicating marked underperformance. The Minimum
Variance and Most Diversified portfolios perform moderately but remain
less efficient than RPP w/o Bonds in managing downside risk.

\begin{table}
\centering
\caption{\label{tab:sharpe-ratios}Sharpe Ratios}
\centering
\begin{tabular}[t]{l|r|r|r|r}
\hline
  & MVP & RPP w/o Bonds & RPP & MDP\\
\hline
StdDev Sharpe (Rf=9.5\%, p=95\%): & -11.911931 & -10.791299 & -15.497496 & -11.796211\\
\hline
VaR Sharpe (Rf=9.5\%, p=95\%): & -8.190001 & -7.082659 & -9.669357 & -7.967569\\
\hline
ES Sharpe (Rf=9.5\%, p=95\%): & -4.003934 & -3.050884 & -3.666896 & -3.660190\\
\hline
SemiSD Sharpe (Rf=9.5\%, p=95\%): & -11.316888 & -10.336831 & -14.644187 & -11.410470\\
\hline
\end{tabular}
\end{table}

\subsubsection{Additional Downside Risk
Measures}\label{additional-downside-risk-measures}

The third table introduces additional downside and opportunity-based
performance metrics, including the Sortino Ratio, conditional drawdown,
upside potential, Expected Shortfall (ES), and Value at Risk (VaR).
These metrics provide a more refined evaluation of portfolio performance
by emphasizing the balance between downside risk and potential gains.

The Sortino Ratio, which measures returns relative to downside risk,
reveals that all portfolios perform similarly, with slight differences
around -0.996. This indicates uniformly weak performance when evaluated
against the MAR of 8.957\%. Conditional drawdown at the 5\% confidence
level highlights the Most Diversified portfolio as the best performer
(0.0397), followed closely by RPP w/o Bonds (0.0400). These results
indicate that these portfolios experience relatively smaller extreme
drawdowns. In contrast, RPP (0.0283) performs the weakest in this
metric, reflecting greater vulnerability to extreme losses. Upside
potential, measured against a MAR of 9\%, is undefined (NaN) for all
portfolios, likely due to the absence of meaningful positive deviations
above the MAR during the evaluation period. For ES, which represents the
average of the worst-case losses, RPP w/o Bonds (-0.0201) achieves the
best performance. The Most Diversified portfolio (-0.0178) and RPP
(-0.0142) exhibit slightly worse outcomes. A similar pattern is observed
for VaR, another tail-risk metric, where RPP w/o Bonds (-0.0123)
performs the best, while RPP (-0.0086) lags behind.

The downside risk metrics such as conditional drawdown favor the Most
Diversified and RPP w/o Bonds portfolios. However, both upside potential
and tail risk metrics highlight significant challenges faced by all
portfolios in achieving favorable risk-adjusted returns during the
evaluation period.

\begin{table}
\centering
\caption{\label{tab:downside-metrics}Downside and Opportunity-based Performance Metrics}
\centering
\begin{tabular}[t]{l|r|r|r|r}
\hline
  & MVP & RPP w/o Bonds & RPP & MDP\\
\hline
Sortino Ratio (MAR = 8.957\%) & -0.9960579 & -0.9952025 & -0.9976658 & -0.9959800\\
\hline
Conditional Drawdown 5\% & 0.0382246 & 0.0400242 & 0.0282992 & 0.0380035\\
\hline
Upside Potential (MAR = 9\%) & NaN & NaN & NaN & NaN\\
\hline
ES & -0.0183472 & -0.0201133 & -0.0142241 & -0.0178732\\
\hline
VaR & -0.0106980 & -0.0123244 & -0.0085753 & -0.0107399\\
\hline
\end{tabular}
\end{table}

To evaluate whether the RPP performs reasonably well across all economic
environments, the returns, volatility, and Sharpe Ratio were calculated
for different economic states. The results in Table 3.5 suggest that
RPPs struggle to consistently deliver reasonable returns in all economic
regimes.

While the RPP w/o Bonds slightly improves performance in high-growth
regimes, with a Sharpe Ratio of 0.0501 compared to 0.0476 for RPP, and
in recessions, with a Sharpe Ratio of 0.0483 compared to 0.0245 for RPP,
both portfolios underperform in these regimes relative to their
performance during moderate growth. In moderate growth conditions, the
Sharpe Ratios of 0.0865 for RPP and 0.0824 for RPP w/o Bonds indicate
more favorable risk-adjusted returns. However, elevated volatility
during recessions, measured at 0.0090 for RPP and 0.0147 for RPP w/o
Bonds, underscores the challenges of maintaining stability during
adverse economic conditions. Although the inclusion or exclusion of
bonds introduces slight performance variations, neither configuration of
the RPP fully meets expectations for producing stable and reasonable
returns across all economic states, particularly during downturns.

\begin{table}
\centering
\caption{\label{tab:regime}Performance at Different Economic States}
\centering
\begin{tabular}[t]{l|l|r|r|r}
\hline
Dataset & regime & mean\_return & volatility & sharpe\_ratio\\
\hline
MV & High Growth & 0.0003836 & 0.0078146 & 0.0490857\\
\hline
MV & Moderate Growth & 0.0005376 & 0.0063563 & 0.0845814\\
\hline
MV & Recession & 0.0006793 & 0.0137366 & 0.0494489\\
\hline
RPP & High Growth & 0.0002948 & 0.0061991 & 0.0475549\\
\hline
RPP & Moderate Growth & 0.0004485 & 0.0052060 & 0.0861512\\
\hline
RPP & Recession & 0.0002215 & 0.0090461 & 0.0244847\\
\hline
RPP\_nb & High Growth & 0.0004408 & 0.0087913 & 0.0501382\\
\hline
RPP\_nb & Moderate Growth & 0.0005836 & 0.0070822 & 0.0824036\\
\hline
RPP\_nb & Recession & 0.0007127 & 0.0147488 & 0.0483259\\
\hline
MD & High Growth & 0.0003763 & 0.0078932 & 0.0476730\\
\hline
MD & Moderate Growth & 0.0006788 & 0.0066212 & 0.1025149\\
\hline
MD & Recession & 0.0010281 & 0.0135732 & 0.0757465\\
\hline
\end{tabular}
\end{table}

\section{Concluding Remarks}\label{concluding-remarks}

This study assessed the feasibility of a South African-based RPP. Two
RPP configurations were analyzed: one including South African stocks and
government bonds with 2- and 10-year maturities, and another excluding
bonds. These portfolios were compared to a MVP and a MMDP, two widely
recognized risk-minimizing approaches derived from MPT. To evaluate
their performance, cumulative and annualized returns were calculated,
alongside downside and drawdown measures to assess risk-adjusted
returns. Additionally, the returns and volatility of the portfolios were
examined across different economic states to evaluate Bridgewater's
claim that RPPs perform well under all conditions.

The findings present mixed conclusions. The results indicate that the
RPP without bonds performs well in managing downside risk, effectively
limiting short-term losses below the target return. However, it performs
poorly in managing drawdown risk, experiencing deeper and more prolonged
declines during adverse market conditions. This highlights the
portfolio's ability to stabilize returns in the short term but also
reveals vulnerabilities to sustained downturns, emphasizing the need for
enhanced resilience strategies to improve recovery from significant
drawdowns. Similarly, cumulative returns do not provide strong support
for either RPP configuration, as both underperform relative to the MVP
and MDP. The inclusion of bonds in the RPP lowers its overall
performance but results in the least volatile portfolio across all
economic regimes.

Ultimately, the relatively low returns of the RPP portfolios highlight
the frequent need for leverage to achieve competitive returns, as
suggested in previous literature (\citeproc{ref-chaves2011risk}{Chaves
\emph{et al.}, 2011}; \citeproc{ref-hurst2010understanding}{Hurst
\emph{et al.}, 2010}). While RPPs demonstrate strengths in managing
downside risk, their underperformance in other areas raises questions
about their feasibility as a robust investment strategy in the South
African context.

\section*{References}\label{references}
\addcontentsline{toc}{section}{References}

\phantomsection\label{refs}
\begin{CSLReferences}{1}{1}
\bibitem[\citeproctext]{ref-bridgewater-2020}
Bridgewater. 2020. \emph{The all weather story}. {[}Online{]},
Available:
\url{https://www.bridgewater.com/research-and-insights/the-all-weather-story}.

\bibitem[\citeproctext]{ref-asness2012leverage}
Asness, C.S., Frazzini, A. \& Pedersen, L.H. 2012. Leverage aversion and
risk parity. \emph{Financial Analysts Journal}. 68(1):47--59.

\bibitem[\citeproctext]{ref-chaves2011risk}
Chaves, D., Hsu, J., Li, F. \& Shakernia, O. 2011. Risk parity portfolio
vs. Other asset allocation heuristic portfolios. \emph{Journal of
Investing}. 20(1):108.

\bibitem[\citeproctext]{ref-choi2024diversified}
Choi, J., Kim, H. \& Kim, Y.S. 2024. Diversified reward-risk parity in
portfolio construction. \emph{Studies in Nonlinear Dynamics \&
Econometrics}. (0).

\bibitem[\citeproctext]{ref-choueifaty2008towards}
Choueifaty, Y. 2008. Towards maximum diversification. \emph{Available at
SSRN 4063676}.

\bibitem[\citeproctext]{ref-chow2014study}
Chow, T.-M., Hsu, J.C., Kuo, L.-L. \& Li, F. 2014. A study of
low-volatility portfolio construction methods. \emph{The Journal of
Portfolio Management}. 40(4):89--105.

\bibitem[\citeproctext]{ref-costa2022data}
Costa, G. \& Kwon, R.H. 2022. Data-driven distributionally robust risk
parity portfolio optimization. \emph{Optimization Methods and Software}.
37(5):1876--1911.

\bibitem[\citeproctext]{ref-damodaran1999estimating}
Damodaran, A. 1999. Estimating risk free rates. \emph{WP, Stern School
of Business, New York}. 20.

\bibitem[\citeproctext]{ref-hurst2010understanding}
Hurst, B., Johnson, B. \& Ooi, Y.H. 2010. Understanding risk parity.
\emph{AQR Capital Management}.

\bibitem[\citeproctext]{ref-james2023semi}
James, N., Menzies, M. \& Chan, J. 2023. Semi-metric portfolio
optimization: A new algorithm reducing simultaneous asset shocks.
\emph{Econometrics}. 11(1):8.

\bibitem[\citeproctext]{ref-loayza2007macroeconomic}
Loayza, N.V., Ranciere, R., Serven, L. \& Ventura, J. 2007.
Macroeconomic volatility and welfare in developing countries: An
introduction. \emph{The World Bank Economic Review}. 21(3):343--357.

\bibitem[\citeproctext]{ref-maillard2010properties}
Maillard, S., Roncalli, T. \& Tiletche, J. 2010. The properties of
equally weighted risk contribution portfolios. \emph{Journal of
Portfolio Management}. 36(4):60.

\bibitem[\citeproctext]{ref-e5a1bb8f-41b7-35c6-95cd-8b366d3e99bc}
Markowitz, H. 1952. Portfolio selection. \emph{The Journal of Finance}.
7(1):77--91. {[}Online{]}, Available:
\url{http://www.jstor.org/stable/2975974} {[}2025, January 24{]}.

\bibitem[\citeproctext]{ref-qian2005risk}
Qian, E. et al. 2005. Risk parity portfolios: Efficient portfolios
through true diversification. \emph{Panagora Asset Management}.

\bibitem[\citeproctext]{ref-qian2011risk}
Qian, E. 2011. Risk parity and diversification. \emph{Journal of
Investing}. 20(1):119.

\bibitem[\citeproctext]{ref-roncalli2013introduction}
Roncalli, T. 2013. \emph{Introduction to risk parity and budgeting}. CRC
Press.

\bibitem[\citeproctext]{ref-stefanova-2020}
Stefanova, K. 2020. \href{}{{The end of the golden era for risk
parity}}. \emph{The Hedgefund Journal}. (149).

\end{CSLReferences}

\bibliography{Tex/ref}





\end{document}
